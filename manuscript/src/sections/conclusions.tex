\section{Conclusions}\label{sec:conclusions}
We conducted a longitudinal thermal comfort study that aimed to develop personal thermal comfort models.
Twenty participants took part in it, and they completed on average at least six right-here-right-now surveys per day for a period of 6 months.
We developed an effective methodology that simplified the life of the participants, and none of them dropped from the study.
We measured and logged environmental parameters, physiological signals, outdoor weather data, and participants' location outdoors and indoors.
We used these data to train and test a personal thermal comfort model for each participant.
We were able to determine that:

\begin{itemize}
    \item Cozie, a micro-EMA open-source Fitbit clock-face, is a reliable and robust solution to non-intrusively collect participants' feedback in field studies.
    \item Personal comfort models were able to accurately predict (median F1-micro score \var{median_f1_score_no_gnb}) occupants' thermal preference.
    With the limitations in data collection posed by the study methodology, they could outperform the PMV model.
    \item \Acf{t-db}, \acf{t-nb}, \acf{hr}, \acf{t-sk-w}, and \acf{w-i}, listed in decreasing order of importance, had the highest average marginal contribution to the overall model prediction.
    \item The thermal personal comfort model prediction accuracy (F1-micro) plateaued at around 300 data points across all participants.
    Individual personal models are sensitive to dataset size to varying degrees.
    The amount of data required to characterize thermal comfort could potentially be reduced with the development of targeted sampling, which strategically requests feedback only when it is necessary.
    \item We made available publicly the data we collected and open-sourced the Python code we used to analyze them to enable other researchers to test different hypotheses utilizing our data.
\end{itemize}

\section*{Acknowledgments}\label{sec:acknowledgments}
This research has been supported by the Republic of Singapore's National Research Foundation through a grant to the Berkeley Education Alliance for Research in Singapore (BEARS) for the Singapore-Berkeley Building Efficiency and Sustainability in the Tropics (SinBerBEST) Program.