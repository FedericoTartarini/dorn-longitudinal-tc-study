\section{Introduction}\label{sec:introduction}
Occupant thermal comfort significantly affects how people perceive their indoor environment, and dissatisfaction is an ongoing challenge.
Evidence shows that approximately 40~\% of 90,000 occupants are dissatisfied with their thermal environment~\cite{Graham2021-en}.
Thermal comfort models are designed to predict comfort towards addressing this challenge.
All major thermal comfort standards have models that are considered aggregate in nature~\cite{iso7730, ASHRAE552020}.
All mainstream aggregate models aim to predict how a ``typical'' person or a group of people would perceive their thermal environment in terms of given environmental (e.g.,\ relative humidity, \ac{t-db}) and personal (i.e.,\ clothing and metabolic rate) parameters.
For example, the \ac{pmv}, the most widely used model in thermal comfort research, predicts the average thermal sensation of a group of people sharing the same environment, as an outcome of the heat transfer balance model between the human body and its surrounding environment.
The \ac{pmv} was developed through laboratory experiments by Fanger~\cite{Fanger1970}, and is now included in both the \gls{7730}~\cite{iso7730} and \gls{55} Standards~\cite{ASHRAE552020}.

\subsection{Limitations of aggregate models}\label{subsec:limitations_aggregate}
Both the \ac{pmv} and the adaptive models have several limitations when used to control the temperature in buildings~\cite{Kim2018a, Xie2020, deDear2013IndoorAir}, despite their successful adoption into international standards.
(1) \textit{Required inputs} -- In real buildings, it is extremely challenging to accurately measure some input variables needed to calculate \ac{pmv}, such as metabolic rate, clothing, airspeed, and mean radiant temperature~\cite{VanHoof2008}.
(2) \textit{Prediction accuracy} -- Even when all input variables are accurately measured, these models have poor accuracy both in predicting group and individual thermal comfort~\cite{Cheung2019}.
(3) \textit{Training} -- Aggregate models do not adapt or re-learn~\cite{Xie2020}.
They were developed using fixed and limited datasets and did not benefit from new feedback provided by people.
They do not learn and adapt to specific conditions~\cite{Kim2018a}.
(4) \textit{Limited inputs} -- Aggregate models only use a small set of input variables.
They do not use variables, such as \ac{t-sk}, \ac{hr}, time of day, age, or health status, that may affect thermal perceptions of people~\cite{Kim2018a}.

\subsection{The emergence of personal comfort models}\label{subsec:personal_models}
Personal comfort models challenge the \emph{one-size-fits-all} approach of aggregate models.
Instead of an average response from a group of people, a single model is trained and tested for each participant.
Personal comfort models are, however, not limited to predicting one person's thermal preference.
Their aggregated outputs can be used to predict the thermal preference of a large group of people sharing the same environment~\cite{Kim2018a}.
Since their introduction, personal comfort models have been expanded to leverage data collected using a wide array of sensors, including portable sensors and devices~\cite{Lee2017-tx, Konis2017-eo}, building management systems~\cite{Lee2019-oj, Guenther2019-hb}, personal comfort systems~\cite{Aryal2020-tp} as well as onboard sensors in wearable devices and smartphones.
This network of sensors can remotely and non-intrusively measure, log and store spatiotemporal environmental and physiological data.

Wearable devices have increased the viability of personal model development due to the use of \emph{physiological sensors} to improve model accuracy.
For example, \acf{t-sk} reflects the vasomotor tone~\cite{Romanovsky2018} while heart rate correlates with activity levels.
This is supported by previous research that has shown that the use of \ac{t-sk} as an independent variable can improve the prediction accuracy of thermal comfort models~\cite{Choi2017, Choi2012a, Liu2019a, Ji2017, Xiong2016, Katic2020-pt}.
In certain applications, \ac{t-sk} may be even determined using non-contact sensors like infrared~\cite{Ghahramani2016, Li2019a, Cosma2018a}.
However, it is essential to emphasize that non-contact sensors are less accurate than those that are in direct contact with the skin;
they can only monitor \ac{t-sk} from body areas that are in the line of sight to the camera and are expensive to install~\cite{Xie2020}.
They, however, do not require having a sensor to be worn by people.
Experimental methodologies collecting \ac{t-sk} are common and \glspl{ib} sensors are often used.
They can accurately measure and log \ac{t-sk}~\cite{VanMarkenLichtenbelt2006, Hasselberg2013a, Tartarini2020d}.
Currently, most smartwatches on the market can measure \ac{hr} with sufficient accuracy for thermal comfort research;
however, none incorporate sufficiently accurate skin temperature sensors~\cite{Liu2019a}.

\subsection{Limitations of personal comfort models}\label{subsec:limitations_pcm}
Despite the momentum of personal comfort models, there are still several unknowns and limitations as outlined in a recent review~\cite{ArakawaMartins2022}.
This analysis pinpoints a lack of diversity in space types, climates, and conditions used to train personal comfort models.
The review showed that only 3 out of 37 studies selected for analysis included data collection outside office spaces or lab-based thermal chambers used to emulate an office environment~\cite{ArakawaMartins2022}.
Another limitation is that there was a wide range of the amount of longitudinal data collected in the studies, with anywhere between 8 and 416 points collected per person.
Researchers placed little emphasis on whether the length and data amount were exhaustive in capturing the predictability of an individual.
In addition, in personal comfort model experiments, it is not common or easy to log and measure information about the participant's dynamic personal factors such as clothing or activity levels~\cite{Ngarambe2019-aw}.
Addressing the lack of diversity and the amount of data is not easy due to experimental constraints.

One of the biggest challenges that researchers currently face is recording how people perceive their thermal environment over a long period of time while minimizing the fatigue of completing a \ac{rhrn} thermal comfort survey.
To partially solve this issue, \textcite{Kim2018} tried to infer occupants' thermal preferences by analyzing specific behaviors, such as turning on and off heating and cooling devices.
They then coupled these data with environmental readings to infer a user's preferences without them having to complete a survey.
However, thermal actions may be triggered by other reasons besides thermal discomfort;
for example, \textcite{Kim2018} found that users turn on the heating element in their chair to mitigate back pain.

\subsection{Improving personal comfort models through larger and more diverse longitudinal data}\label{subsec:our_approach}
To address the limitations mentioned above, an emerging methodology focuses on the use of wearable devices to collect physiological data and act as the subjective feedback collection interface. 
This method builds upon research in the area of Ecological Momentary Assessments (EMA), a form of collecting subjective information in diverse field-based setting~\cite{Shiffman2008-xb}. 
A style of this methodology emerging as a popular way to reduce the incidence of survey fatigue is micro-EMA, in which smartwatches are used to prompt a research participant to leave feedback in a fast and time-efficient manner~\cite{Intille2016-xb}.
Micro-EMA has been shown to deliver higher response rates with a lower burden on research participants than a smartphone or computer-based survey~\cite{Ponnada2017-dl}.
To build upon this foundation and help solve the issue of collecting perception data from people, our team has contributed to the development of the micro-EMA Cozie project that targets indoor occupant data collection~\cite{Jayathissa2019a}.
Cozie is an open-source application that one can install on Fitbit (Versa 2 and Ionic) or Apple smartwatches.
The platform has been utilized in previous studies to test the implementation and modeling of smartwatch-based subjective data collection~\cite{Jayathissa2020-pv, Quintana2021-ka}, study the thermal preference, imbalanced classes~\cite{Quintana2020-zu}, and create personal comfort models using building information model components as inputs~\cite{Abdelrahman2021-mx}.
One can find more information about Cozie and the official documentation at \url{https://cozie.app} and \url{https://cozie-apple.com}.
Cozie allows people to conveniently complete a \ac{rhrn} survey via their smartwatches.
Subjects' perceptions, preferences, and behaviors collected via Cozie can then be coupled with environmental data collected from wireless sensing devices and physiological data collected by the smartwatch.

\subsection{Aim and objectives}\label{subsec:aim_objectives}
Our research aims to resolve gaps in personal thermal comfort models by collecting field-based thermal preference data.
Our methodology is designed to enable us to address the following questions with resulting novel insights:

\begin{itemize}
    \item How many data points per user must be collected to develop a reliable and robust personal comfort model?
        This study seeks to collect more than twice the amount of data per person compared to previous studies.
        We collected data for 180 days resulting in more feedback responses per person (up to 1080) than in any previous study~\cite{ArakawaMartins2022}.
    \item Are environmental and physiological data sufficient to create personal thermal comfort models while minimizing the impact on users? 
        The methodology of this paper utilizes a novel framework of simple-to-use non-intrusive techniques to collect physiological, environmental, and geospatial data using smartwatch-based micro-EMA.
    \item Can increasing diversity of space types and conditions improve the accuracy of personal comfort models?
        How can different variables contribute to the overall model accuracy?
        This study is designed to collect data from diverse spaces, including the participants' homes, where there is a lack of data in previous studies.
        In addition, this paper is novel in accurately monitoring whether the \ac{rhrn} was completed during transitory conditions.
\end{itemize}

In addition, we decided to publicly share our data so other people can use it to test different hypotheses or develop personal comfort models using a different methodology.